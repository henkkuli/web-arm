\documentclass{beamer}

\usepackage{amsmath}
\usepackage[utf8]{inputenc}
\usepackage[T1]{fontenc}
\usepackage{parskip}
\usepackage{graphicx}
\usepackage{epstopdf}
\usepackage{lmodern,textcomp}
\usepackage[main=finnish,english]{babel}   
\usepackage{tikz}
\usepackage[european,cute inductors]{circuitikz}
%\usepackage{a4wide}            % Eihän tässä työssä ollut mitään sivurajoituksia?
\usepackage{todonotes}

\usepackage{parskip}
\usepackage[section]{placeins}
%\usepackage{comment}
%\usepackage{titling}
\usepackage{url}
\usepackage{listingsutf8}
\usepackage{listings}
\usepackage{booktabs}
\usepackage{siunitx}
\usepackage{minted}

\author{Henrik Lievonen}
\title{ARM-assemblyn alkeet}

\begin{document}

\frame{\titlepage}

% Mikä assembly
\frame{
    \frametitle{Sisällys}
    \tableofcontents
}

\section{Mikä ihmeen assembly}
\begin{frame}[fragile]
    \frametitle{Mikä ihmeen assembly}
    \begin{columns}[T]
        \begin{column}{.48\textwidth}
            \begin{minted}{c++}
int arr[10] = {
    0, 1, 2, 3, 4,
    5, 6, 7, 8, 9
};
int s = 0;
for (int i = 0; i < 10; i++)
    s += arr[i];
            \end{minted}
        \end{column}
        \pause
        \begin{column}{.48\textwidth}
            \begin{minted}{arm}
    mov r5, #0
    mov r4, #0
        b   .L2
.L3:
    ldr r3, .L5
    ldr r3, [r3, r4, lsl #2]
    add r5, r5, r3
    add r4, r4, #1
.L2:
    cmp r4, #9
    ble .L3

.L5: .word arr
arr: .word 0, 1, 2, 3, 4
     .word 5, 6, 7, 8, 9
            \end{minted}
        \end{column}
    \end{columns}
\end{frame}

\section{Mikä ihmeen ARM}
\begin{frame}
    \frametitle{Mikä ihmeen ARM}
    \begin{itemize}
        \item \emph{Advanced RISC Machine} (aiemmin \emph{Acorn RISC Machine})
        \item 32-bittinen mikroprosessoriarkkitehtuuri
        \item 16 yleisrekisteriä (\mintinline{arm}{r0} - \mintinline{arm}{r15})
        %\item Yksi vanhimmista RISC-arkkitehtuureista
        \item Suosittu pienissä, sulautetuissa järjestelmissä
        \item Vähävirtainen
        \item Tarvittaessa hyvinkin yksinkertainen
        %\item Brittiläisen Acornin 80-luvulla kehittämä

    \end{itemize}
\end{frame}

% Mikä arm
\section{RISC vs CISC}
\begin{frame}
    \frametitle{RISC vs CISC}
    \centering
    \begin{tabular}{p{.45\textwidth} | p{.45\textwidth}}
        RISC & CISC \\
        \hline
        \foreignlanguage{english}{\emph{Reduced Instruction Set Computer}} & \foreignlanguage{english}{\emph{Complex Instruction Set Computer}} \\
        Käskyt tekevät yhden asian & Käskyt voivat tehdä monta asiaa \\
        Käskyt tasamittaisia & Käskyt vaihtelevan mittaisia \\
        Suurin osa käskyistä vakioaikaisia & Eri käskyt vievät eri verran aikaa \\
        \emph{ARM}, \emph{MIPS}, \emph{SPARC}, \emph{Alpha}, \emph{PowerPC} & \emph{x86}, \emph{VAX}, \emph{IBM S/360}, \emph{PDP-11} \\
    \end{tabular}
\end{frame}

\begin{frame}[fragile]
    \frametitle{RISC vs CISC}
    \begin{columns}[T]
        \begin{column}{.45\textwidth}
            ARM:
            \begin{minted}{arm}
    mov r5, #0
    mov r4, #0
        b   .L2
.L3:
    ldr r3, .L5
    ldr r3, [r3, r4, lsl #2]
    add r5, r5, r3
    add r4, r4, #1
.L2:
    cmp r4, #9
    ble .L3

.L5: .word arr
arr: .word 0, 1, 2, 3, 4
     .word 5, 6, 7, 8, 9
            \end{minted}
        \end{column}
        \begin{column}{.51\textwidth}
            x86\_64:
            \begin{minted}{gas}
    movl    $0, %r12d
    movl    $0, %ebx
    jmp .L2
.L3:
    movslq  %ebx, %rax
    movl    arr(,%rax,4), %eax
    addl    %eax, %r12d
    addl    $1, %ebx
.L2:
    cmpl    $9, %ebx
    jle .L3

arr: .word 0, 1, 2, 3, 4
     .word 5, 6, 7, 8, 9
            \end{minted}
        \end{column}
    \end{columns}
\end{frame}
% Mikä RISC
% RISC vs CISC

\section{Komennot}
\begin{frame}[fragile]
    \frametitle{Komennot}
    \begin{minted}{arm}
start:
add    r0, r1, r2
subeq  r0, r1, r2
mulseq r0, r1, r2
eors   r0, r1, #10
rsb    r0, r1, r2, lsl #2
and    r0, r1, r2, asr r3
mov    r0, r1
cmp    r0, r1, ror #15
b      start
bl     start
ldr    r0, [r1, #-4]
ldr    r0, [r1, r2]
ldr    r0, [r1, r2, lsl #2]
    \end{minted}
\end{frame}

% Komentoluokat
\section{Lisää ARM-arkkitehtuurista}
% Ehdollisuus
\begin{frame}
    \frametitle{Lisää ARM-arkkitehtuurista}
    Ehdolliset komennot
    \centering
    \begin{tabular}{p{.15\textwidth} | p{.75\textwidth}}
        \hline
        eq & Yhtäsuuri kuin \\
        ne & Erisuuri kuin \\
        cs / hs & Etumerkitön suurempi tai yhtäsuuri kuin \\
        cc / lo & Etumerkitön pienempi kuin \\
        mi & Negatiivinen \\
        pl & Positiivinen tai nolla \\
        vs & Ylivuoto \\
        vc & Ei ylivuotoa \\
        hi & Etumerkitön suurempi kuin \\
        ls & Etumerkitön pienempi tai yhtäsuuri kuin \\
        ge & Etumerkillinen suurempi tai yhtäsäsuuri kuin \\
        lt & Etumerkillinen pienempi kuin \\
        gt & Etumerkillinen suurempi kuin \\
        le & Etumerkillinen pienempi tai yhtäsuuri kuin \\
        al & Aina \\
    \end{tabular}
\end{frame}
% Siirrot
\begin{frame}
    \frametitle{Lisää ARM-arkkitehtuurista}
    Siirrot
    \centering
    \begin{tabular}{p{.15\textwidth} | p{.75\textwidth}}
        \hline
        lsr & Looginen siirto oikealle \\
        lsl & Looginen siirto vasemmalle \\
        asr & Aritmeettinen siirto oikealle \\
        ror & Pyöritys oikealle \\
        rrx & Pyöritys oikealle muistinumerolla \\
    \end{tabular}
    
    Parametrit
    \begin{itemize}
        \item Vakioiden oltava oikealla välillä ($\approx 0 - \approx 32$)
        \item Rekistereistä käytetään alinta 8 bittiä
    \end{itemize}
\end{frame}
% CPSR
\begin{frame}
    \frametitle{Lisää ARM-arkkitehtuurista}
    \begin{itemize}
        \item CPSR
        \begin{itemize}
            \item \emph{Current Program Status Register}
            \item Suorittimen nykyinen tila
            \item ALUn tila (ehtoja varten)
            \item Onko keskeytykset käytössä
        \end{itemize}
    \end{itemize}
\end{frame}
% Erikoismerkitykset (konventioita)
\begin{frame}
    \frametitle{Lisää ARM-arkkitehtuurista}
    \begin{itemize}
        \item Rekisterit
        \begin{itemize}
            \item \mintinline{arm}{r15} program counter \mintinline{arm}{pc}
            \item \mintinline{arm}{r14} link register \mintinline{arm}{lr}
            \item \mintinline{arm}{r13} stack pointer \mintinline{arm}{sp}
            \item \mintinline{arm}{r12} Intra-Procedure-call scratch register
            \item \mintinline{arm}{r11} frame pointer \mintinline{arm}{fp}
            \item \mintinline{arm}{r4} - \mintinline{arm}{r10} local variables
            \item \mintinline{arm}{r0} - \mintinline{arm}{r3} arguments and return values
        \end{itemize}
    \end{itemize}
\end{frame}
% Eri tilat (ehkä)
\section{Esimerkkejä}
\begin{frame}
    \frametitle{Esimerkkejä}
\end{frame}
% Peruskomentojen syntaksi
% Pieniä ohjelmia


\end{document}

